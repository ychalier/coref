\documentclass{beamer}
\usepackage[utf8]{inputenc}
\usepackage{sansmathaccent}
\pdfmapfile{+sansmathaccent.map}
\usepackage{tikz}

\usetheme{Singapore}

\title{Co-reference resolution}
\author{Yohan Chalier}\institute{Symbolic Natural Language Processing (SD213)}

\begin{document}

\begin{frame}
\titlepage
\end{frame}

\begin{frame}{What is co-reference?}

\emph{Co-reference} occurs when two or more expressions refers to the same referent. Usually, one expression is in a full form (the \emph{antecedent}) and the other one in a abbreviated form (a \emph{proform}).

~\par

For example:

\textit{The \textbf{music} was so loud that \textbf{it} couldn't be enjoyed.}

~\par

Co-reference resolution is needed to derive a correct interpretation of a text.

\end{frame}

\begin{frame}{The problem of co-reference resolution}

\begin{block}{Naive algorithm}
Look for the nearest preceding individual that is compatible with the referring expression.
\end{block}

~\par

It solves sentences like this:

{\em \hspace{\parindent} The girl\textsubscript{1} likes her\textsubscript{1} brother\textsubscript{2} and protects him\textsubscript{2}.}

~\par

But it fails to differentiate those sentences:

{\em \hspace{\parindent} He\textsubscript{?} said that John\textsubscript{?} was coming.}

{\em \hspace{\parindent} His\textsubscript{1} sister said that John\textsubscript{1} was coming.}

\end{frame}

\begin{frame}{Domination and c-command}

\begin{block}{Domination}
Node $N_1$ dominates node $N_2$ if $N_1$ is above $N_2$ in the tree and one can trace a path from $N_1$ to $N_2$ moving only downwards in the tree (never upwards).
\end{block}

\begin{block}{c-command}
Node $N_1$ c-commands node $N_2$ if
\begin{itemize}
\item $N_1$ does not dominate $N_2$
\item $N_2$ does not dominate $N_1$
\item The first (i.e. the lowest) branching node that dominates $N_1$ also dominates $N_2$
\end{itemize}
\end{block}

\end{frame}

\begin{frame}{Domination and c-command}

\begin{center}
\begin{tikzpicture}
\node{M}
	child {node {A}}
	child {node {B}
		child {node {C} child {node {E}}}
		child {node {D}
			child {node {F}}
			child {node {G}}
		}
	};
\end{tikzpicture}
\end{center}

\end{frame}

\begin{frame}{Co-reference and c-command}

It was hypothesized that one restriction between proform and antecedent is that {\bfseries the proform cannot appear in a position where it \emph{c-commands} its antecedent}.

~\par

This is not trivial: Bouchard, Denis. (2010). \textit{Une explication cognitive des effets attribués à la c-commande dans les contraintes sur la coréférence}. Corela. 10.4000/corela.965.

\end{frame}


\end{document}